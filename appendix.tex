\chapter{Παράδειγμα Παραρτήματος}

% \section{Πρώτη ενότητα}
% Τα συστήματα ομότιμων κόμβων, προκειμένου να υποστηρίζουν πιο
% εκφραστικές λειτουργίες αναπαράστασης και αναζήτησης δεδομένων,
% εξελίχθηκαν στα συστήματα ομότιμων κόμβων τα οποία βασίζονται στις
% τεχνολογίες του Σημασιολογικού Ιστού για την αναπαράσταση των
% δεδομένων μέσω σχημάτων που τα περιγράφουν (\en{Schema-based
% peer-to-peer systems}).

% Συμπερασματικά το σύστημα που αναπτύχθηκε στα πλαίσια αυτής της
% διπλωματικής είναι ένα πλήρες σύστημα ομότιμων κόμβων βασισμένο σε
% σχήματα, το οποίο καθιστά δυνατή την αναζήτηση της πληροφορίας με
% ένα διαφορετικό τρόπο απ' ότι τα προϋπάρχοντα  συστήματα.

% \section{Μελλοντικές Επεκτάσεις}
% Το σύστημα που αναπτύχθηκε στα πλαίσια αυτής της διπλωματικής
% εργασίας θα μπορούσε να βελτιωθεί και να επεκταθεί περαιτέρω,
% τουλάχιστον ως προς τρεις κατευθύνσεις. Συγκεκριμένα, αναφέρονται
% τα ακόλουθα:

% \begin{itemize}
% \item Ενσωμάτωση διαδικασίας επιλογής σχήματος με βάση το οποίο ο
% κόμβος θα συμμετέχει στο σύστημα. Έτσι όπως έχει σχεδιαστεί το
% σύστημα, κάθε κόμβος έχει τη δυνατότητα να δημιουργήσει πολλά
% σχήματα και να αποθηκεύσει δεδομένα σε περισσότερα από ένα. Ως
% σχήμα του κόμβου (με βάση το οποίο απαντάει τις ερωτήσεις),
% θεωρείται το τελευταίο στο οποίο αποθήκευσε δεδομένα. Η δυνατότητα
% επιλογής θα του παρείχε περισσότερη ευελιξία.
% \item Δυνατότητα αντιστοίχισης δεδομένων τα οποία να μην είναι
% αποθηκευμένα σε βάση δεδομένων αλλά σε αρχεία. Η αποδέσμευση από
% τη βάση δεδομένων θα έκανε το σύστημα πιο εύκολο στην εγκατάσταση
% και τη χρήση.
% \item Αξιολόγηση του συστήματος ως προς τη συμπεριφορά του αν
% συμμετέχει σε αυτό μεγάλος αριθμός κόμβων \en{(scalability
% testing)} και αν χρησιμοποιηθεί ένα πολύ μεγάλο καθολικό σχήμα. H
% αξιολόγηση αυτή αφορά την ταχύτητα με την οποία ένας κόμβος
% παίρνει απαντήσεις σε μια ερώτηση καθώς και την ποιότητα των
% απαντήσεων.
% \end{itemize}

% %
% 	\begin{table}[!tb]
% 		\centering
% 		\caption{Πίνακας αλήθειας της λογικής συνάρτησης \en{F}}
% 		\small
% 		\renewcommand{\arraystretch}{1.3}
% 		\begin{tabular}{| c | c | c || c |}
% 			\hline               
% 		  	\textbf{\en{A}} & \textbf{\en{B}} &  \textbf{\en{C}} &   \textbf{\en{F}} \\
% 			\hline
% 				  0 & 0 & 0 & 0  \\
% 				  0 & 0 & 1 & 0  \\
% 				  0 & 1 & 0 & 1  \\
% 				  0 & 1 & 1 & 0  \\	
% 				  1 & 0 & 0 & 1  \\
% 				  1 & 0 & 1 & 0  \\
% 				  1 & 1 & 0 & 1  \\
% 				  1 & 1 & 1 & 0  \\
% 		  	\hline
% 		\end{tabular}
% 		\label{tableAppA.01}
% 	\end{table}
% %