\begin{abstract}
Στη σύγχρονη εποχή, οι ψηφιακές τηλεπικοινωνίες αποτελούν μια τεχνολογία που αποκτά συνεχώς περισσότερες πρακτικές εφαρμογές (\en{LTE-A, Wifi, DVB-S}). Αυτό οφείλεται, σε σημαντικό βαθμό, στη χρήση τεχνικών κωδικοποίησης καναλιού. Μέσω αλγορίθμων επαναληπτικής αποκωδικοποίησης (\en{iterative decoding algorithms}), έχει καταστεί δυνατή η λειτουργία των σύγχρονων τηλεπικοινωνιακών συστημάτων κοντά στο όριο χωρητικότητας, οδηγώντας σε γρήγορη και αξιόπιστη επικοινωνία. Το αποτέλεσμα είναι η χρήση τους να γίνεται ολοένα και πιο διαδεδομένη, σε πληθώρα εφαρμογών.

Στόχος της διπλωματικής αυτής εργασίας, είναι ο προγραμματισμός των \en{LDPC Repeat-Accumulate} κωδίκων με βάση το πρότυπο \en{DVB-S2} σε \en{Matlab}, η καταγραφή και η παρουσίαση των επιδόσεών τους, καθώς και η πρόταση για περαιτέρω διερεύνησή τους μέσω προγραμματισμού σε \en{GPU} με βάση την αρχιτεκτονική \en{CUDA}.
\end{abstract}

\begin{abstracteng}
\tl{In the modern age, digital communications constitute a technology with an increasing number of practical applications (LTE-A, Wifi, DVB-S). This is due, to a significant extent, to the utilization of channel coding schemes. Through iterative decoding algorithms, it has been made possible for modern digital communication systems to operate close to their capacity βοθνδ, leading to fast and reliable communication. As a result, their use is being more and more wide, in a multitude of applications.} 

\tl{This Master thesis aims to the programming in Matlab of the LDPC Repeat-Accumulate codes described in the DVB-S2 standard, to the measurement and presentation of their performance and to the proposition of their further investigation via their programming on a GPU, by using the CUDA architecture.}

\end{abstracteng}
