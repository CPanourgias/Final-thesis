\begin{abstract}
Στη σύγχρονη εποχή, οι ψηφιακές τηλεπικοινωνίες αποτελούν ένα εργαλείο που αποκτά συνεχώς περισσότερες πρακτικές εφαρμογές (\en{LTE-A, Wifi, DVB-S}). Αυτό οφείλεται, σε σημαντικό βαθμό, στη χρήση τεχνικών κωδικοποίησης καναλιού. Μέσω αλγορίθμων επαναληπτικής αποκωδικοποίησης (\en{iterative decoding algorithms}), έχει καταστεί δυνατή η λειτουργία των σύγχρονων τηλεπικοινωνιακών συστημάτων κοντά στο όριο χωρητικότητας, οδηγώντας σε γρήγορη και αξιόπιστη επικοινωνία. Το αποτέλεσμα είναι η χρήση τους να γίνεται ολοένα και πιο διαδεδομένη, σε πληθώρα εφαρμογών.

Στόχος της διπλωματικής αυτής εργασίας, είναι ο προγραμματισμός ενός \en{LDPC Repeat-Accumulate} κώδικα με βάση το πρότυπο \en{DVB-S2} σε \en{Matlab}, η καταγραφή και η παρουσίαση των επιδόσεων του, καθώς και η πρόταση για περαιτέρω δυνατότητες του μέσω προγραμματισμού σε \en{GPU} με βάση την αρχιτεκτονική \en{CUDA}.
\end{abstract}

\begin{abstracteng}
\tl{In the modern age, digital communications consist a tool that constantly acquires more and more practical application (LTE-A, Wifi, DVB-S). This is due, to a significant extent, to the utilization of channel coding schemes. Through iterative decoding algorithms, it has been made possible for modern digital communication systems to operate close to their capacity level, leading to fast and reliable communication. As a result, their use is being more and more wide, in a multitude of applications} 

\tl{This diploma thesis aims to the development of an LDPC Repeat-Accumulate code, based on the DVB-S2 protocol in Matlab, to the recording and presentation of its performance and to the proposition of further capabilities via its development on a GPU, using the CUDA architecture.}

\end{abstracteng}