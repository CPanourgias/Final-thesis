\chapter{\selectlanguage{greek}Mελλοντικές επεκτάσεις}
Έχει ήδη αναφερθεί σε προηγούμενο κεφάλαιο, η έντονη επιδεκτικότητα για παραλληλοποίηση, όσον αφορά την αποκωδικοποίηση των \en{LDPC} και κατ'επέκταση των \en{RA} κωδίκων. Καθώς οι ασύρματες συσκευές μεταδίδουν και δέχονται δεδομένα υψηλού ρυθμού σε πραγματικό χρόνο, αυξάνεται ραγδαία η ανάγκη για ταχύτητα και αξιοπιστία στην επικοινωνία.

Η χρήση των \en{LDPC} σε πολλά νέα πρότυπα (\en{DVB-S2, WiMAX (802.16e), Wifi (802.11n), 10 Gbit Ethernet (802.3an)}  κ.λ.π. \cite{falcao2011massively}, καθώς και σε πολλαπλούς ρυθμούς δεδομένων για τα διαφορετικά πρότυπα, έχουν κάνει ορατή τη δυσκολία για υλοποίηση \en{hardware}, καθώς η επεξεργασία βασικής ζώνης στο επίπεδο του φυσικού εξαρτήματος είναι πολυδάπανη σε εύρος ζώνης και επεξεργαστική ισχύ. 

Αντ'αυτού, ο σχεδιασμός συστημάτων ψηφιακής τηλεπικοινωνίας υιοθετεί όλο και περισσότερο υλοποιήσεις σε \en{software}, μέσω της χρήσης \en{CPUs} ή/και \en{GPUs}, για την υλοποίηση των επικοινωνιακών προτοκόλλων \cite{park2011parallel}, \cite{abburi2011scalable}. Σχετική έρευνα, έχει δείξει τη διαφορά στη χρήση \en{GPU} σε σχέση με κυκλώματα \en{ASIC} για \en{LDPC} αποκωδικοποίση \cite{falcao2009gpus}.

\section{Προσομοίωση σε \en{CUDA}}

Η χρήση προγραμματισμού σε \en{GPU} προσφέρει υψηλή υπολογιστική ισχύ, καθώς οι μονάδες επεξεργασίας γραφικών αποκτούν ολοένα και καλύτερες επιδόσεις. Προς την κατεύθυνση αυτή, αξιοσημείωτο ενδιαφέρον παρουσιάζει ο προγραμματισμός με βάση την αρχιτεκτονική \en{Copmpute Unified Device Architecture (CUDA)} της εταιρίας \en{nVidia}.

Σχετική έρευνα έχει υπάρξει αναφορικά με τη χρήση \en{GPU} για τη διαχείρηση του \en{SPA} και την εξαγωγή πληροφορίας από τις μετρικές \en{LLR} \cite{falcao2009parallel}. Ακόμη, υπάρχει προτεινόμενος τρόπος για την υλοποίηση \en{CUDA} για χρήση \en{LDPC} αποκωδικοποίησης \cite{falcao2011massively}, καθώς και υβριδική συνδυαστική χρήση \en{multicore CPU - GPU}, για την ενσωμάτωση πολλαπλών ρυθμών και τηλεπικοινωνιακών προτύπων \cite{park2011parallel}.

Με βάση, λοιπόν, τα παραπάνω αποκτά ενδιαφέρον η μελέτη προς την κατεύθυνση της ταχύτερης υλοποίησης \en{RA} αποκωδικοποιητών κάνοντας χρήση της αρχιτεκτονικής \en{CUDA}, καθώς και η μέτρηση και παρουσίαση των επιδόσεων τους.