\chapter{\selectlanguage{greek}Συμπεράσματα και μελλοντικές επεκτάσεις}

\section{Προσομοίωση σε \en{CUDA}}
% Έχει ήδη αναφερθεί σε προηγούμενο κεφάλαιο, η έντονη επιδεκτικότητα για παραλληλοποίηση, όσον αφορά την αποκωδικοποίηση των \en{RA} κωδίκων. Οι τεχνολογίες ψηφιακής τηλεπικοινωνίας, έχουν γνωρίσει ραγδαία ανάπτυξη τα τελευταία χρόνια και οι προσπάθειες επικεντρώνονται στ

% % Digital mobile communication technologies, such as next generation mobile communication and mobile TV, are
% % rapidly advancing. Hardware designs to provide baseband processing of new protocol standards are being actively
% % attempted, because of concurrently emerging multiple standards and diverse needs on device functions, hardware-
% % only implementation may have reached a limit. To overcome this challenge, digital communication system designs
% % are adopting software solutions that use central processing units or graphics processing units (GPUs) to implement
% % communication protocols. In this article we propose a parallel software implementation of low density parity check
% % decoding algorithms, and we use a multi-core processor and a GPU to achieve both flexibility and high
% % performance. Specifically, we use OpenMP for parallelizing software on a multi-core processor and Compute
% % Unified Device Architecture (CUDA) for parallel software running on a GPU. We process information on H-matrices
% % using OpenMP pragmas on a multi-core processor and execute decoding algorithms in parallel using CUDA on a
% % GPU. We evaluated the performance of the proposed implementation with respect to two different code rates for
% % the China Multimedia Mobile Broadcasting (CMMB) standard, and we verified that the proposed implementation
% % satisfies the CMMB bandwidth requirement.