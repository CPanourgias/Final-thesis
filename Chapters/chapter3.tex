\chapter{\selectlanguage{greek}Κώδικες Επανάληψης-Συσσώρευσης \en{(repeat - accumulate, RA)}}
\externaldocument{chapter1.tex}
\externaldocument{chapter2.tex}
Στο κεφάλαιο αυτό, παρουσιάζονται οι κώδικες Επανάληψης-Συσσώρευσης \en{(repeat - accumulate, RA)}, η μελέτη της επίδοσης των οποίων αποτελεί και το αντικείμενο της παρούσας εργασίας. Επισημαίνεται πως το κανάλι μετάδοσης είναι διακριτό \en{AWGN} κανάλι και πως το σώμα που υποθέτουμε είναι το $\mathbb{F}_2$.

Οι \en{RA} αποτελούν τους πρώτους κώδικες που βασίζονται σε συσσωρευτές και εφευρέθηκαν από τον \en{D. Divsalar} κ.ά. \cite{divsalar1998coding}. Ενώ έχουν απλή δομή, δακρίνονται από καλές επιδόσεις, κυρίως στην κατεύθυνση της πρακτικής κωδικοποίησης \en{LDPC} κωδίκων, των οποίων αποτελούν υποομάδα. Η αποκωδικοποίησή τους μπορεί να αντιμετωπισθεί είτε ως σειριακή \en{turbo}, είτε ως \en{LDPC}, τακτική που είναι και πιο ευρέως διαδεδομένη \cite{ryan2009channel}.

Στη συνέχεια του κεφαλαίου παρουσιάζονται οι κώδικες που μπορούν να πλησιάσουν το θεωρητικό όριο χωρητικότητας που επιβάλει το θεώρημα \en{Shannon} (Θεώρημα \ref{theorem:shannon}), διατηρώντας τη δυνατότητα πρακτικής αποκωδικοποίησης. Αφού γίνει μια σύντομη αναφορά στη σχέση χωρητικότητας και σηματοθορυβικής σχέσης (\en{SNR}), παρουσιάζονται οι τρόποι διαχείρησης των \en{RA}, είτε ως \en{turbo}, είτε ως \en{LDPC}. Κατόπιν αφού αναλυθεί ο τρόπος κωδικοποίησης των \en{LDPC} (ως τον πιο συχνά χρησιμοποιούμενο τρόπο αντιμετώπισης των \en{RA}), διατυπώνεται πλήρως το \en{coding scheme} των \en{RA}, που δίνει και τη βάση πάνω στην οποία στηρίζεται η προσομοίωση που έγινε και θα αναλυθεί στο Κεφάλαιο 4.

\section{Κώδικες που πλησιάζουν τη χωρητικότητα}
\en{\lipsum[1]}

\subsection{Η χωρητικότητα ως \en{SNR}}
% Έστω τηλεπικοινωνιακό σύστημα, με ζωνοπερατό κανάλι, εύρους ζώνης $W$, που εισάγει \en{AWGN} θόρυβο, με φασματική πυκνότητα ισχύος $N_0/2$. Η χωρητικότητα καναλιού αποδεικνύεται ότι μπορεί να οριστεί από την παρακάτω εξίσωση (Θεώρημα \en{Shannon - Hartley}):

% \begin{equation}
% C=W\cdot log_2(1+ \frac{S}{N})
% \label{eq:capacity}
% \end{equation}
% όπου, $S$ η λαμβανόμενη ισχύς του μεταδιδόμενου σήματος στο δέκτη, $N$ η ισχύς θορύβου και $W$ το εύρος ζώνης του καναλιού. Η ισχύς του θορύβου δίνεται από τη σχέση:

% \begin{equation}
% N = \int_{-W}^{W}\frac{N_0}{2}\,df=N_0\cdot{W}.
% \label{eq:Noise}
% \end{equation}
% ,ενώ η συματοθορυβική σχέση του συστήματος (\en{signal-to-noise ratio - SNR}) ορίζεται ως:

% \begin{equation}
% SNR=\frac{S}{N}
% \label{eq:SNR}
% \end{equation}

% Από τις \ref{eq:capacity} και \ref{eq:SNR} και κανονικοποιώντας ως προς το εύρος ζώνης $W$, προκύπτει:

% \begin{equation}
% C=log_2(1+SNR)
% \label{eq:capacity vs SNR}
% \end{equation}
\en{\lipsum[1]}
\section{\en{RA} ιδωμένοι ως \en{turbo}}
\en{\lipsum[1]}
\section{\en{RA} ιδωμένοι ως \en{LDPC}}
\en{\lipsum[1]}
% We shall consider only binary LDPC codes for the sake of simplicity, although
% LDPC codes can be generalized to nonbinary alphabets as is done in Chapter
% 14. A low-density parity-check code is a linear block code given by the null spaceof an m × n parity-check matrix H that has a low density of 1s. A regular LDPC
% code is a linear block code whose parity-check matrix H has column weight g and
% row weight r, where r = g(n/m) and g  m. If H is low density, but its row and
% column weight are not both constant, then the code is an irregular LDPC code.
% For irregular LDPC codes, the various row and column weights are determined by
% one of the code-design procedures discussed in subsequent chapters. For reasons
% that will become apparent later, almost all LDPC code constructions impose the
% following additional structural property on H: no two rows (or two columns) have
% more than one position in common that contains a nonzero element. This property
% is called the row–column constraint, or simply, the RC constraint.
% The descriptor “low density” is unavoidably vague and cannot be precisely
% quantified, although a density of 0.01 or lower can be called low density (1% or
% fewer of the entries of H are 1s). As will be seen later in this chapter, the density
% need only be sufficiently low to permit effective iterative decoding.


\subsection{Κωδικοποίηση \en{LDPC}}
\en{\lipsum[1]}
\subsection{Κωδικοποίηση \en{RA}}
\en{\lipsum[1]}
\subsection{Αποκωδικοποίηση \en{RA}}
\en{\lipsum[1]}







% \begin{itemize}

% \item ο κωδικοποιητής πρέπει να αποθηκεύσει $2^k$ κωδικές λέξεις μήκους $n$ σε ένα \
% enquote*{κωδικό λεξικό} (η πολυπλοκότητα περιγραφής του κώδικα είναι εκθετική ως προς το \en{n})


% \item ο κωδικοποιητής πρέπει, επιπλέον, να αποθηκεύσει την αντιστοίχιση μεταξύ των μηνυμάτων και των κωδικών λέξεων


% \item ο αποκωδικοποιητής πρέπει μετά την απόφασή του για την αποσταλθείσα κωδική λέξη, να αναζητήσει στο προηγούμενο λεξικό, 
% το μήνυμα που της αντιστοιχεί.
% \

% end{itemize}



